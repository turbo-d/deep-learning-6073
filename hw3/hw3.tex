\documentclass{article}
\usepackage{caption}
\usepackage{graphicx}

\title{CS 6073 Homework 3: Medical Image Segmentation}
\author{
    Luginbuhl, Dale \\
    \texttt{luginbdr@mail.uc.edu}
}
\date{October 30th 2023}


\begin{document}

\maketitle

\section{Data}
\subsection{How many data samples are included in the dataset?}
The dataset is pre-split into a train dataset and a test dataset. The train dataset
contains 80 samples. The test dataset contains 20 samples.

\subsection{Which problem will this dataset try to address?}
The goal is to predict the segmentation of retina blood vessels from
high-resolution retinal fundus images. ``Accurate segmentation of
blood vessels is a critical task in ophthalmology as it aids in
the early detection and management of various retinal pathologies"

\subsection{What is the dimension ranging in the dataset?}
Each data sample has an input image of size 3x512x512 (CHW) pixels, for
a total of 786,432 feature dimensions. Each feature (pixel) takes on
integer values in the range $[0,255]$.
\vspace{5mm}

\noindent Each data sample has a label image of size 1x512x512 (CHW)
pixels, for a total of 262,144 feature dimensions. Each feature (pixel) takes
on integer values in the range $[0,1]$. Values of 0 represent background pixels, while
values of 1 represent blood vessel pixels.

\subsection{Does the dataset have any missing information? E.g., missing features.}
The dataset does not have any missing information.

\subsection{What is the label of this dataset?}
The label of the dataset is a mask image representing the segmentation of
retina blood vessels. The mask images are of size 1x512x512 (CHW) pixels, and
the pixels take on integer values in the range $[0,1]$. Values of 0 represent
background pixels, while values of 1 represent blood vessel pixels.

\subsection{How many percent of data will you use for training, validation and testing?}
The dataset is pre-split into 80 training samples and 20 testing samples. We will
further split the training set into a training set and a validation set. The validation
set will consists of 20 samples and the remaining training set will consist of 60 samples.

The resulting percentage splits will then be:
\begin{description}
    \item Training : 60\%
    \item Validation : 20\%
    \item Test : 20\%
\end{description}

\subsection{What kind of data pre-processing will you use for your training dataset?}
Besides converting the image tensors to floating point values, no data pre-processing
was performed. With more time we would've explored data augmentation, as the data set
was limited in size, and even the 2-layer unet models were likely complex enough to
overfit.


\section{Model}

\begin{table}[h]
\begin{center}
\begin{tabular}{c|c|c}
    Model & Dice & IoU \\
    \hline
    U-Net 2 layers & 0.0010 & 22.3292 \\
    U-Net 3 layers & 0.0010 & 22.3292 \\
    U-Net 4 layers & 0.0010 & 22.3292
\end{tabular}
\end{center}
\caption{Comparison of model architectures}
\label{table:models}
\end{table}

\section{Objective}
We chose dice loss as the loss function to train the model. We chose
this instead of cross-entropy because it is more robust to the class
imbalance in the segmentation masks. There are many more background
pixels than blood vessel pixels.

\section{Optimization}
We chose the Adam optimizer to take advantage of momentum in our optimization.
Adam allows us to converge to an optimum value more quickly, and it allows
provides the potential to escape from local minima.

\section{Model Selection}
\begin{table}[h]
\begin{center}
\begin{tabular}{c|c|c}
    Model & Dice \& IoU w/ norm & Dice \& IoU w/o norm \\
    \hline
    U-Net 2 layers & dice 0.0010, IoU 22.3292 & dice 0.0010, IoU 22.3292 \\
    U-Net 3 layers & dice 0.0010, IoU 22.3292 & dice 0.0010, IoU 22.3292 \\
    U-Net 4 layers & dice 0.0010, IoU 22.3292 & dice 0.0010, IoU 22.3292
\end{tabular}
\end{center}
\caption{Model selection}
\label{table:model_selection}
\end{table}

\subsection{How do you avoid overfit your model and underfit your model?}
We could apply L1 or L2 regularization, dropout, or perform data augmentation.

\section{Model Performance}

In Figures 1 through 6 you can see the loss, dice, and IoU plots for the
training and validation data sets for each model. Or well really, you probably
can't see anything because the plots are too small. Don't worry though, you aren't
missing any meaningful information. The data in the Figures, as well as in Tables
1 and 2 came from real training runs, but they don't tell us anything other than
that the training was not being performed correctly. Unfortunately, running short
on both time and compute, I was unable to debug any further. The ``4 input image
with label and models prediction" was not included as the images were useless
given the invalid models. The code to generate the image \textit{was} written
though and can be found in \textit{generate\_prediction\_image.py}.

\begin{figure}[h]
    \centering
    \includegraphics[width=1.0\textwidth]{./unet_2_layer/train_val_plots.png}
    \caption{U-Net 2 layers training and validation Dice and IoU}
    \label{fig:unet_2_layer}
\end{figure}

\begin{figure}[h]
    \centering
    \includegraphics[width=1.0\textwidth]{./unet_2_layer_batch_norm/train_val_plots.png}
    \caption{U-Net 2 layers with batch norm training and validation Dice and IoU}
    \label{fig:unet_2_layer_batch_norm}
\end{figure}

\begin{figure}[h]
    \centering
    \includegraphics[width=1.0\textwidth]{./unet_3_layer/train_val_plots.png}
    \caption{U-Net 3 layers training and validation Dice and IoU}
    \label{fig:unet_3_layer}
\end{figure}

\begin{figure}[h]
    \centering
    \includegraphics[width=1.0\textwidth]{./unet_3_layer_batch_norm/train_val_plots.png}
    \caption{U-Net 3 layers with batch norm training and validation Dice and IoU}
    \label{fig:unet_3_layer_batch_norm}
\end{figure}

\begin{figure}[h]
    \centering
    \includegraphics[width=1.0\textwidth]{./unet_4_layer/train_val_plots.png}
    \caption{U-Net 4 layers training and validation Dice and IoU}
    \label{fig:unet_4_layer}
\end{figure}

\begin{figure}[h]
    \centering
    \includegraphics[width=1.0\textwidth]{./unet_4_layer_batch_norm/train_val_plots.png}
    \caption{U-Net 4 layers with batch norm training and validation Dice and IoU}
    \label{fig:unet_4_layer_batch_norm}
\end{figure}

\end{document}